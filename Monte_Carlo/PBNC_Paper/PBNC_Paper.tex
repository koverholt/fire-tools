% No compression; prevents errors when viewing PDF on Windows
\pdfobjcompresslevel=0

\documentclass[12pt]{article}

\usepackage{mathptmx,times}
\usepackage{amsmath}
\usepackage{tabularx}
\usepackage{booktabs, ragged2e, array, dcolumn}
\usepackage[font=small,format=plain,labelfont=bf,up]{caption}
\usepackage[bookmarks,hidelinks]{hyperref} % Enables table of content in PDF viewers; hide link boxes

\setlength{\textwidth}{6.5in}
\setlength{\textheight}{8.5in}
\setlength{\topmargin}{0in}
\setlength{\headheight}{0.25in}
\setlength{\headsep}{0.5in}
\setlength{\oddsidemargin}{0in}
\setlength{\evensidemargin}{0.0in}
\setlength{\voffset}{-0.4in}
\setlength{\footskip}{0.5in}
\setlength{\marginparpush}{7pt}
\setlength{\parskip}{\baselineskip}
\setlength{\parindent}{0pt}

\usepackage{titlesec}
\usepackage{graphicx}
\usepackage{floatrow}

\titlelabel{\thetitle.\quad}
\titleformat*{\section}{\bf\normalsize}
\titleformat*{\subsection}{\bf\normalsize}
\titlespacing\section{0pt}{12pt plus 4pt minus 2pt}{0pt plus 2pt minus 2pt}
\titlespacing\subsection{0pt}{12pt plus 4pt minus 2pt}{0pt plus 2pt minus 2pt}
\titlespacing\subsubsection{0pt}{12pt plus 4pt minus 2pt}{0pt plus 2pt minus 2pt}

\begin{document}

\begin{flushright}
\textbf{PBNC2014-218}
\end{flushright}

\begin{center}
\textbf{COMBINED FIRE MODEL UNCERTAINTY AND INPUT PARAMETER UNCERTAINTY ANALYSIS FOR NUCLEAR POWER PLANT FIRE SCENARIOS}
% \textbf{A MONTE CARLO ANALYSIS OF THE EFFECT OF FIRE SEVERITY ON NUCLEAR SAFETY STRUCTURES, SYSTEMS, AND COMPONENTS}
\end{center}

\begin{center}
\textbf{M. Clouthier\textsuperscript{1}, K. Overholt\textsuperscript{2}}\\

\textsuperscript{1}Clouthier Risk Engineering, Nova Scotia, Canada\\
\textsuperscript{2}National Institute of Standards and Technology, Gaithersburg, MD, USA
\end{center}

\begin{center}
\textbf{Abstract}
\end{center}

Quantitative fire risk analysis is used in conducting safety assessments at nuclear facilities to evaluate the potential consequences of postulated fire scenarios.

In fire modelling applications, the key phenomena that influence the predicted figure of merit (i.e., the specific objective of the analysis) are known; however, in some cases the existing state of knowledge and the adequacy of existing modelling tools to predict a given phenomenon is in need of improvement. It is therefore necessary to consider relevant uncertainties in a rational quantitative manner.

In this paper, a Monte Carlo analysis is applied to a fire scenario involving potential damage to nuclear safety structures, systems, and components. The effect of deterministic peak heat release rate and fire growth on the predicted hot gas layer is studied. The results of the case study are presented to demonstrate an improved method to develop a design basis fire.

\section{Introduction}
\label{sec:introduction}
% !TEX root = PBNC_Paper.tex

%	Statements about the field of research to provide the reader with a setting or context for the problem to be investigated and to claim its centrality or importance.

The area of uncertainty analysis is well established in science and engineering \cite{Morgan}. 
Its role is to provide insight on the impact of uncertainties associated with engineering evaluations. This allows for meaningful and defensible decision making.

Generally, there are three types of uncertainty associated with model prediction: Parameter uncertainty, model uncertainty, and completeness uncertainty. 

%	More specific statements about the aspects of the problem already studied by other researchers, laying a foundation of information already known.

Uncertainty is addressed in fire protection engineering literature \cite{Notarianni:SFPE}. 
And treatment of uncertainty analysis is explicitly required in most technical standards related to fire safety at nuclear reactor facilities \cite{NFPA:805, NUREG:6850}. 
Model bias and uncertainty has been quantified in Nuclear Regulatory Commission (NRC) NUREG-1824~\cite{NUREG_1824_Sup_1} for a number of fire models which are commonly used in nuclear power plant applications. This information can be used as part of a specific methodology to evaluate model uncertainty  that is presented in NRC NUREG-1934~\cite{NUREG_1934}. 

%	Statements that indicated the need for more investigation, creating a gap or research niche for the present study to fill.

The practice of uncertainty analysis in fire modelling is evolving and the methodology employed currently is left to the practitioners' discretion. In many cases, a bounding analysis approach is taken, based on the invalid assumption that uncertainty is addressed by specifying conservative values for input parameters. Bounding analysis by adopting a series of conservative assumptions as a substitute for uncertainty analysis could result in overly conservative decisions (at best) or provide a false understanding of the actual margin of safety.

Although treatment of fire model uncertainty and input parameter sensitivity is clearly addressed in NRC NUREG-1934, parameter uncertainty is not specifically addressed. Given that greatest uncertainty associated with fire protection engineering calculations stems from input parameters, a clear methodology would be useful for practitioners.

%	Statements giving the purpose/objectives of the writer's study or outlining its main activity or findings.

In this paper, we demonstrate the calculation of three cases: 1) the effect of model bias and uncertainty, 2) the effect of input parameter uncertainty, and 3) the combined effect of model bias/uncertainty and input parameter uncertainty.

%	Optional statements that give a positive value or justification for carrying out that study.

This paper provides a straight-forward and practical example of uncertainty analysis for a zone fire modelling application. 

\section{Previous work}
\label{sec:previous_work}

% !TEX root = PBNC_Paper.tex

This section discusses previous work on sensitivity and uncertainty.











\section{Fire model scenario and computational setup}
\label{sec:fire_model_scenario_setup}

The Consolidated Model of Fire Growth and Smoke Transport (CFAST)~\cite{CFAST_Users_Guide_6} zone model was used. The ``Cabinet Fire in Switchgear Room'' scenario in Appendix B of NRC NUREG-1934~\cite{NUREG_1934} was selected for this uncertainty analysis. The compartment has dimensions of 26.5~m by 18.5~m by 6.1~m. The ambient temperature was specified as 20~$^\circ$C. The simulation time is 1~h (3600~s). The wall and ceiling materials were specified as concrete with a thermal conductivity of 1.6~W/m-K, density of 2400~kg/m$^3$, and specific heat of 0.75~J/g-K. For the purposes of this uncertainty analysis, the fire was considered as the input parameter of interest and was specified as a constant HRR, which is described in more detail in the following section.

The Python programming language was used as a wrapper to automate CFAST model runs, collect the output from the CFAST model, run the Monte Carlo simulation, calculate probabilities of exceeding the threshold temperature, and generate the plots. Python is a high-level, flexible, and powerful programming language that is well-suited to scientific and engineering applications~\cite{Oliphant:2007}. Additional modules that were used with Python include NumPy~\cite{oliphant2006guide}, SciPy~\cite{Jones:2001fk}, and matplotlib~\cite{Hunter:2007}. The source code of the scripts that were used for these examples is open source and freely available for download.\footnote{\url{https://code.google.com/p/fire-tools/}}


\section{Uncertainty analysis}
\label{sec:uncertainty_analysis}

 To simplify this analysis and to present a clear example, the HRR was selected as the only model input parameter of interest. In reality, there is some amount of uncertainty associated with other input parameters. The input distribution for the HRR is assumed to be a gamma distribution from NUREG 6850~\cite{NUREG_6850}, with a 75th percentile HRR of 232~kW and a 98th percentile HRR of 1002~kW. This distribution has a shape parameter $k$ of 0.46 and a scale parameter $\theta$ of 386. This corresponds to the case of vertical cabinets with unqualified cable, fire in more than one cable bundle open doors. The model output quantity of interest is the HGL temperature. A threshold (or critical) HGL temperature of 100~$^\circ$C was used in this analysis.

Three cases are presented that account for different levels of uncertainty: Case 1 accounts for model bias and uncertainty, Case 2 accounts for input parameter uncertainty, and Case 3 accounts for both model bias/uncertainty and input parameter uncertainty. For Case 1, a HRR of 1002~kW is considered, which is the 98th percentile HRR value. For Cases 2 and 3, the gamma distribution is used directly. These three cases are described in more detail in the following sections.



\clearpage


\subsection{Case 1: Effect of model bias and uncertainty}

This case only accounts for model bias and uncertainty. A HRR of 1002~kW was selected as the input fire size, which represents the 98th percentile HRR. The probability density function (PDF) and cumulative distribution function (CDF) of the input HRR distribution are shown in Fig.~\ref{fig:case_1_input_distributions}. The probability of exceeding the threshold HGL temperature can be calculated following the procedure described in NUREG~1934~\cite{NUREG_1934}.

\textbf{Step 1}: Specify the input HRR and run CFAST to calculate the resulting HGL temperature.

For an input HRR of 1002~kW, the CFAST model predicts an HGL temperature of 90.7~$^\circ$C.

\textbf{Step 2}: Subtract the ambient value of the HGL temperature (20~$^\circ$C) to determine the predicted temperature rise.
\begin{equation}
M = 90.7~^\circ\textrm{C} - 20~^\circ\textrm{C} = 70.7~^\circ\textrm{C}
\end{equation}

\textbf{Step 3}: Referring to Table XX of NUREG 1824 Supplement 1~\cite{NUREG_1824_Sup_1}, which indicates that, on average, CFAST overpredicts HGL temperatures in forced ventilation scenarios with a model bias factor $\delta$ of 1.15. The adjusted model prediction is calculated as
\begin{equation}
\mu = \frac{M}{\delta} = \frac{70.7~^\circ\textrm{C}}{1.15} \approx 81.5~^\circ\textrm{C}
\end{equation}
Referring again to Table XX of NUREG 1824 Supplement 1~\cite{NUREG_1824_Sup_1}, the model relative standard deviation $\widetilde\sigma_M$ is 0.20. The standard deviation of the distribution is calculated as
\begin{equation}
\sigma = \widetilde\sigma_M \left( \frac{M}{\delta} \right) = 0.20 \left( \frac{70.7}{1.15} \right) \approx 12.3~^\circ\textrm{C}
\end{equation}
Figure~\ref{fig:case_1_output_distributions} shows the PDF and CDF of the adjusted HGL temperature distribution.

\textbf{Step 4}: Calculate the probability that the actual HGL temperature will exceed 100~$^\circ$C as
\begin{equation}
\textrm{P}(T > 100~^\circ\textrm{C}) = \frac{1}{2} \textrm{erfc} \left( \frac{T - \mu}{\sigma \sqrt{2}} \right) = \frac{1}{2} \textrm{erfc} \left( \frac{100~^\circ\textrm{C} - 81.5~^\circ\textrm{C}}{12.3~^\circ\textrm{C} \sqrt{2}} \right) \approx 0.07
\end{equation}
For Case 1, the probability of exceeding the threshold HGL temperature of 100~$^\circ$C was calculated as 0.07. Note that this estimate is based only on model bias and uncertainty.


\clearpage


\begin{figure}[p]
\includegraphics[width=2.6in]{Figures/input_PDF_point}
\includegraphics[width=2.6in]{Figures/input_CDF_point}
\caption{PDF and CDF of input HRR distribution; case with model bias and uncertainty.}
\label{fig:case_1_input_distributions}
\end{figure}

\begin{figure}[p]
\includegraphics[width=2.6in]{Figures/output_PDF_1_model}
\includegraphics[width=2.6in]{Figures/output_CDF_1_model}
\caption{PDF and CDF of output HGL temperature distribution; case with model bias and uncertainty.}
\label{fig:case_1_output_distributions}
\end{figure}


\clearpage


\subsection{Case 2: Effect of input parameter uncertainty}

This case only accounts for input parameter uncertainty. The probability of exceeding the threshold HGL temperature can be calculated following the procedure described in NUREG~1934~\cite{NUREG_1934}.

\textbf{Step 1}: Specify the input HRR distribution.

The input distribution for the HRR is assumed to be a gamma distribution from NUREG 6850~\cite{NUREG_6850}, with a 75th percentile HRR of 232~kW and a 98th percentile HRR of 1002~kW. This corresponds to the case of vertical cabinets with unqualified cable, fire in more than one cable bundle open doors. Other distributions could be used to represent the amount of uncertainty in the input parameter, such as a uniform or normal distribution.

The Monte Carlo method can be used to propagate the distribution of the input parameter (HRR) through a fire model to calculate the resulting distribution of output quantities (HGL temperature). For this case, 50,000 Monte Carlo iterations were performed. The PDF and CDF of the input HRR distribution are shown in Fig.~\ref{fig:case_2_input_distributions}.

\textbf{Step 2}: Draw a random sample from input HRR distribution and run CFAST to calculate the HGL temperature. The HGL temperature value is stored, then we return to Step~1 to continue drawing samples from the HRR distribution until the specified number of Monte Carlo iterations is reached.

Figure~\ref{fig:case_2_output_distributions} shows the PDF and CDF of the resulting HGL temperature distribution.

\textbf{Step 3}: Calculate the probability that the HGL temperature will exceed the threshold HGL temperature. Because the output HGL distribution was calculated using the Monte Carlo method, we can numerically integrate over the resulting distribution to determine the probability of exceeding the threshold HGL temperature.

For Case 2, the probability of exceeding the threshold HGL temperature of 100~$^\circ$C was calculated as XX. Note that this estimate is based only on input parameter uncertainty.


\clearpage


\begin{figure}[p]
\includegraphics[width=2.6in]{Figures/input_PDF}
\includegraphics[width=2.6in]{Figures/input_CDF}
\caption{PDF and CDF of input HRR distribution; case with input parameter uncertainty.}
\label{fig:case_2_input_distributions}
\end{figure}

\begin{figure}[p]
\includegraphics[width=2.6in]{Figures/output_PDF_2_input}
\includegraphics[width=2.6in]{Figures/output_CDF_2_input}
\caption{PDF and CDF of output HGL temperature distribution; case with input parameter uncertainty.}
\label{fig:case_2_output_distributions}
\end{figure}


\clearpage


\subsection{Case 3: Effect of combined model bias/uncertainty and input parameter uncertainty}

This case accounts for both model bias/uncertainty and input parameter uncertainty. Therefore, this case uses a combination of steps from Cases 1 and 2.

\textbf{Step 1}: Specify the input HRR distribution.

The input distribution for the HRR is the same as in Case 2, a gamma distribution from NUREG 6850~\cite{NUREG_6850}, with a 75th percentile HRR of 232~kW and a 98th percentile HRR of 1002~kW. As in Case 2, 50,000 Monte Carlo iterations were performed. The PDF and CDF of the input HRR distribution are shown in Fig.~\ref{fig:case_3_input_distributions}.

\textbf{Step 2}: Draw a random sample from input HRR distribution and run CFAST to calculate the HGL temperature.

\textbf{Step 3}: For the sample in Step~2, subtract the ambient value of the HGL temperature (20~$^\circ$C) to determine the predicted temperature rise.

\textbf{Step 4}: Referring to Table XX of NUREG 1824 Supplement 1~\cite{NUREG_1824_Sup_1}, which indicates that, on average, CFAST overpredicts HGL temperatures in forced ventilation scenarios with a model bias factor $\delta$ of 1.15, and the adjusted model prediction is calculated. Referring again to Table XX of NUREG 1824 Supplement 1~\cite{NUREG_1824_Sup_1}, the model relative standard deviation $\widetilde\sigma_M$ is 0.20, and the standard deviation of the distribution is calculated.

A random HGL temperature value from the resulting normal distribution is drawn. This HGL temperature value is stored, then we return to Step~1 to continue drawing samples from the HRR distribution until the specified number of Monte Carlo iterations is reached. Figure~\ref{fig:case_3_output_distributions} shows the PDF and CDF of the resulting HGL temperature distribution.

\textbf{Step 5}: Calculate the probability that the HGL temperature will exceed the threshold HGL temperature. Because the output HGL distribution was calculated using the Monte Carlo method, we can numerically integrate over the resulting distribution to determine the probability of exceeding the threshold HGL temperature.

For Case 3, the probability of exceeding the threshold HGL temperature of 100~$^\circ$C was calculated as XX. Note that this estimate is based on the combined effect of model bias and uncertainty.


\clearpage


\begin{figure}[p]
\includegraphics[width=2.6in]{Figures/input_PDF}
\includegraphics[width=2.6in]{Figures/input_CDF}
\caption{PDF and CDF of input HRR distribution; case with combined model bias/uncertainty and input parameter uncertainty.}
\label{fig:case_3_input_distributions}
\end{figure}

\begin{figure}[p]
\includegraphics[width=2.6in]{Figures/output_PDF_3_combined}
\includegraphics[width=2.6in]{Figures/output_CDF_3_combined}
\caption{PDF and CDF of output HGL temperature distribution; case with combined model bias/uncertainty and input parameter uncertainty.}
\label{fig:case_3_output_distributions}
\end{figure}


\clearpage


\section{Conclusions}
\label{sec:conclusions}


\bibliographystyle{unsrt}
\bibliography{references_ko,references_mpc}

\end{document}